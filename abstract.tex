\documentclass{article}
\usepackage{fullpage}
\usepackage{comment}
\usepackage[square, numbers]{natbib}
\usepackage{graphicx}
\usepackage[subrefformat=parens,labelformat=parens]{subfig}
\usepackage{acronym}
\date{\vspace{-5ex}}
\begin{document}


\title{The Risky Business of Asking For Help}
\subtitle{An Agent Based Model of Unmet Need}

\begin{abstract}
The proportion of the UK who are elderly, and require support is increasing rapidly. Identifying those in need of support, and effectively managing the necessarily limited resources available to provide it is a substantial, and growing challenge. This is compounded by the social stigma attached to age, and the need for assistance, which together with a lack of belief that help will be forthcoming can make those most in need of support reluctant to request it.


% We draw together data from multiple survey sources to parameterise, and validate the model.

This poster will outline our present work on developing an agent based model grounded in theories of decision making to explore the help-seeking behaviour of elderly people requiring support with activities of daily living.  
The decision to request help is construed as a balance between severity of need, belief that support will actually be available, and the desire to maintain an individual's self image as a capable person. Paired with this is the challenging task of making the most effective use of available resources to deliver care to those most in need.

This will allow the exploration of the process of help-seeking by and identification of, those in need. with the intention of validating model outputs against longitudinal survey data (e.g. the English Longitudinal Survey of Ageing). In future, the simulation will support investigation of the implications of policy interventions, for example public campaigns to reduce the stigma associated with age.
\end{abstract}

\end{document}